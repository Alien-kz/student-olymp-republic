\begin{enumerate}
\item а) Вычислить предел:
$$\lim_{n \to \infty} \sin^2 \left( \pi \sqrt{n^2 + 2n}\right).$$

б) Зная, что $\int\limits_{0}^{1} 
\frac{\ln(1+x)}{x} dx = \frac{\pi^2}{12}$, вычислить
$$\int\limits_{0}^{1} 
\frac{\ln(1-x^3)}{x} dx.$$

\item Числа 53295, 67507, 88825, 81719, 39083 делятся на 3553. Не вычисляя определитель матрицы $$A = 
\begin{pmatrix}
5 & 3 & 2 & 9 & 5 \\
6 & 7 & 5 & 0 & 7 \\
8 & 8 & 8 & 2 & 5 \\
8 & 1 & 7 & 1 & 9 \\
3 & 9 & 0 & 8 & 3 \\
\end{pmatrix},
$$
доказать, что он делится на 3553.

\item Доказать тождество:
$$\frac{x}{1} + 
\frac{x^3}{1 \cdot 3 }+
\frac{x^5}{1 \cdot 3 \cdot 5} + \ldots = 
\exp^{x^2/2} \cdot \int_{0}^{x} \exp^{-\frac{t^2}{2}} dt.$$

\item Даны две строки, представляющие числа $A$ и $B$ в фибоначчиевой системе счисления. Описать программу, которая находит строку, представляющую число $A + B$ в фибоначчиевой системе счисления. Числа $A$ и $B$ могут превышать максимальное допустимое значение в стандартных целочисленных типах.

Примечание: Числа Фибоначии $F_1$, $F_2$, $\ldots$ определяются начальными значениями:
$$F_1 = 1, F_2 = 2, F_{N+1} = F_N + F_{N-1}.$$
Рассмотрим фибоначчиеву систему счисления с двумя цифрами 0 и 1, в которой, в отличие от двоичной системы весами являются не степени двойки 1, 2, 4, 8, 16, \ldots, а числа Фибоначчи 1, 2, 3, 5, 8, 13, \ldots. В этой системе счисления каждое положительное целое число единственным образом представляется в виде строки нулей и единиц, которая начинается с 1 и в которой нет двух единиц, стоящих рядом.

\item Совсем недавно Вася занялся программированием и решил реализовать собственную программу для игры в шахматы. Но у него возникла проблема определения правильности хода конем, который делает пользователь. Т.е. если пользователь вводит значение <<C7-D5>>, то программа должна определить это как правильный ход, если же введено <<E2-E4>>, то ход неверный. Так же нужно проверить корректность записи ввода: если например, введено <<D9-N5>>, то программа должна определить данную запись как ошибочную. Помогите ему осуществить эту проверку!

Входные данные:

В единственной строке входного файла {\tt INPUT.TXT} записан текст хода (непустая строка), который указал пользователь. Пользователь не может ввести строку, длиннее 5 символов.

Выходные данные:

В выходной файл {\tt OUTPUT.TXT} нужно вывести <<YES>>, если указанный ход конем верный, если же запись корректна (в смысле правильности записи координат), но ход невозможен, то нужно вывести <<NO>>. Если же координаты не определены или заданы некорректно, то вывести сообщение <<ERROR>>.
\end{enumerate}