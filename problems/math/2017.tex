\begin{enumerate}

\item Пусть $(T_n)_{n=1}^{\infty}$ --- последовательность натуральных чисел, заданная рекуррентно: $T_1 = T_2 = T_3 = 1$ и $T_{n+3} = T_{n+2} + T_{n+1} + T_n$ при $n \geqslant 1$. Вычислите сумму ряда $$\displaystyle \sum\limits_{n=1}^{\infty} \frac{T_n}{2^n},$$
если известно, что данный ряд сходится.

\item Найдите все простые $p$, запись которых в $k$-ичной системе счисления при некотором натуральном $k > 1$ содержит ровно $k$ различных цифр (старшая цифра не может быть нулём).

\item Докажите, что в любой группе квадрат произведения двух элементов порядка два и куб произведения двух элементов порядка три всегда являются коммутаторами.

\item Точка $P$ лежит внутри выпуклой области, ограниченной параболой $y = x^2$, но не лежит на оси $OY$. Обозначим через $S(P)$ множество всех точек, полученных отражением $P$ относительно всех касательных к параболе.

а) Докажите, что значение суммы 
$$\displaystyle\max_{(x, y) \in S(P)} y ~ + \min_{(x, y) \in S(P)} y$$ не зависит от выбора точки $P$.

б) Найдите геометрическое место точек $P$ таких, что $\displaystyle\max_{(x, y) \in S(P)} y = 0.$

\item Для каждой функции $f: [0, 1] \to \mathbb{R}$ обозначим через $s_n(f)$ и $S_n(f)$ нижнюю и верхнюю суммы Дарбу для функции $f$, соответствующие равномерному разбиению $[0, 1]$ на $n$ частей. Существует ли такая интегрируемая функция $f$, что $\displaystyle \sum_{n=1}^{\infty} s_n(f)$ сходится, а
$\displaystyle\sum_{n=1}^{\infty} S_n(f)$ расходится?

\item Некоторые участники математической олимпиады списали решения некоторых задач у своих товарищей. Докажите, что можно с позором выгнать часть участников так, чтобы получилось, что более четверти от общего числа списанных решений было списано выгнанными участниками у не выгнанных.

\end{enumerate}
