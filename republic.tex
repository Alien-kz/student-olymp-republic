\documentclass[12pt,a4paper]{article}
\usepackage[utf8]{inputenc}
\usepackage[T2A]{fontenc}
\usepackage[russian]{babel}
\usepackage{amsmath}
\usepackage{amsfonts}
\usepackage{amssymb}
\usepackage[left=1.5cm,right=1.5cm,top=2cm,bottom=2cm]{geometry}

\newcommand{\head}[5]{ %hoster city date
	\addcontentsline{toc}{subsection}{#1 (#5)}
	\begin{center}
	#1 \\
	#2 \\
	#3 \\
	#4 #5\\
	\end{center}
}
\newcommand{\result}{
	\begin{center}
	\textbf{Результаты}
	\end{center}
}

\begin{document}

\newpage

\addcontentsline{toc}{section}{О республиканской олимпиаде}

Республиканская студенческая предметная олимпиада проводится ежегодно весной по поручению министерства образования и науки Республики Казахстан. Олимпиада по каждой специальности проводится отдельно. В данной брошюре собраны олимпиады по двум специальностям: <<Математика>>, <<Математическое и компьютерное моделирование>>. Впервые такая олимпиада прошла в 2009 году на базе Казахского национального университета имени аль-Фараби (г. Алматы). Вплоть до 2015 года олимпиада не меняла свое место проведения по этим направлениям. В 2016 году олимпиады по данным направлениям прошли в Казахстанского филиале МГУ имени М.~В.~Ломомносова (г. Астана), что стало традицией передавать право проведения олимпиады университету, представитель которого стал победителем в личном зачете.


Задачи на олимпиадах, как правило, доступны хорошо подготовленным студентам 2-4 курса. Желающим углубить свои навыки в решении студенческих олимпиадных задач рекомендуется список литературы:

\begin{enumerate}
\item Садовничий В.А., Подколзин А.С. Задачи студенческих олимпиад по математике. М.: Наука. 1978. --- 208 с.
\item Садовничий В.А., Григорьян А.А., Конягин С.В. Задачи студенческих математических олимпиад. Издательство МГУ, 1987. --- 309 с.
\item Gelca R., Andreescu T. Putnam and Beyond. 2007. --- 850 pp.
\item Меньшиков Ф. Олимпиадные задачи по программированию. 2007. --- 320 с.
\end{enumerate}


\newpage

\tableofcontents

\newpage

\addcontentsline{toc}{section}{Республиканская студенческая олимпиада по математике}

\head{VI олимпиада}{Казахский национальный университет имени аль-Фараби}{Алматы}{27 марта}{2014}
\begin{enumerate}
\item Вычислить предел:
$$\lim_{x \to +\infty} \frac{(x - \sqrt{x^2 - 1})^n + (x + \sqrt{x^2 - 1})^n}{x^n}.$$

\item Решить систему уравнений:
$$
\begin{cases}
\frac{x^2}{y^2} + \frac{y^2}{z^2} + \frac{z^2}{x^2} = 3 \\
x + y + z = 3 \\
\frac{y^2}{x^2} + \frac{z^2}{y^2} + \frac{x^2}{z^2} = 3 \\
\end{cases}
$$

\item Дан многочлен с целыми коэффициентами. В трех целых точках он принимает значение 2. Докажите, что ни в какой целой точке он не принимает значение 3.

\item Найти все решения дифференциального уравнения второго порядка $$f''(x) + f(\pi - x) = 1.$$

\end{enumerate}

\result
\begin{center}
\begin{tabular}{|l|l|l|c|c|c|c|c|c|c|}
\hline
№ & Участник & ВУЗ & Курс & 1 & 2 & 3 & 4 & $\Sigma$ & Диплом \\
\hline
1 & Шокетаева Надира & КФ МГУ & 1 & 15 & 25 & 35 & 15 & 90 & 1 степени \\
\hline
2 & Кенжебаева Фариза & АГУ & 2 & 15 & 25 & 35 & 10 & 85 & 2 степени \\
\hline
3 & Тубалыков Кайрат & КФ МГУ & 2 & 15 & 25 & 35 & 10 & 80 & 2 степени \\
\hline
4 & Тыныштыкбай Абылай & КарГУ & 2 & 14 & 23 & 35 & 5 & 77 & 3 степени \\
\hline
5 & Кабак Мухтар & КазНУ & 2 & 15 & 25 & 15 & 0 & 55 & 3 степени \\
\hline
6 & Тилеуова Жаннат & КазНУ & 1 & 15 & 13 & 25 & 0 & 53 & 3 степени \\
\hline
\end{tabular}
\end{center}

%http://www.kaznu.kz/ru/5723
%www.kaznu.kz/content/files/pages/folder3657/матем1.doc
%https://iqaa.kz/images/otchety/inst_accreditation/2014_rus/AGU_IA_2014_rus.pdf
\newpage


\head{VII олимпиада}{Казахский национальный университет имени аль-Фараби}{Алматы}{26 марта}{2015}
\begin{enumerate}
\item Функция $f(x)$ дважды непрерывно дифференцируема на отрезке $[a, b]$ и имеет на $[a, b]$ не менее трех различных нулей. Докажите, что существует точка $x \in [a, b]$ такая, что $f(x) + f''(x) = 2 f'(x)$. 

\item Функция $f(x)$ дважды непрерывно дифференцируема на полуоси $[0, +\infty)$. Известно, что $f(x) > 0$, $f'(x) > 0$ и $\frac{f(x) f''(x)}{(f'(x))^2} \leqslant 2$ для всех $x \in [0, +\infty)$. Доказать, что $\lim_{x \to +\infty} \frac{f'(x)}{(f(x))^2} = 0$.

\item Известно, что для квадратных матриц $A$ и $B$ одинакового порядка выполнены следующие равенства $AB = BA$, $A^{99} = E$, $B^{100} = E$, где $E$ --- единичная матрица. Докажите, что существует матрица $C$ такая, что $(A + E + B) C = C (A + E + B) = E$.

\item Пусть
$$A = 
\begin{pmatrix}
3 & 2 \\
2 & 0 \\
\end{pmatrix},
A^n =
\begin{pmatrix}
a_{11}(n) & a_{12}(n) \\
a_{21}(n) & a_{22}(n) \\
\end{pmatrix}.
$$

Доказать существование и найти предел $\lim_{n \to \infty} \frac{a_{12}(n)}{a_{22}(n)}$.

\end{enumerate}

\result
\begin{center}
\begin{tabular}{|l|l|l|c|c|c|c|c|c|c|}
\hline
№ & Участник & ВУЗ & Курс & Диплом \\
\hline
1 & Шокетаева Надира & КФ МГУ & 2 & 1 степени \\
\hline
2 & Журавская Александра &  КФ МГУ & 1 & 2 степени \\
\hline
3 & Тыныштыкбай Абылай & КарГУ & 3 & 2 степени \\
\hline
4 & Джумагулов Серик & КарГУ & ? & 3 степени \\
\hline
5 & Кабак Мухтар & КазНУ & 3 & 3 степени \\
\hline
6 & Таскынов Ануар & КФ МГУ & 2 & 3 степени \\
\hline
\end{tabular}
\end{center}

%http://www.kaznu.kz/content/files/news/folder14971/%D0%9E%D0%BB%D0%B8%D0%BC%D0%BF%D0%B8%D0%B0%D0%B4%D0%B0_%D0%BE%D1%82%D1%87%D0%B5%D1%82%202015.doc
%http://student.ksu.kz/novosti/student-fakulteta-matematiki-i-informacionnyh-tehnologii-kargu-im-e-a-buketova-tynyshtykbai-abylai-zanjal-2-e-i-dzhumagulov-serik-zanjal-3-e-mesto-v-lichnom-zachete-na-vii-respublikanskoi-stud.html

\newpage


\head{VIII олимпиада}{Казахстанский филиала МГУ имени М.В.Ломоносова}{Астана}{1 апреля}{2016}
\begin{enumerate}

\item Какие натуральные числа представимы в виде $x^2-y^2+2x+2y$ для некоторых целых $x$ и $y$?

\item Пусть  $\alpha(x)$ --- первая цифра после запятой в десятичной записи числа $2^x$.\\
а) Докажите, что функция $\alpha(x)$ интегрируема по Риману на $[0, 1]$.\\
б) Докажите, что $\displaystyle 3.5 < \int\limits_{0}^{1} \alpha(x) dx < 4.5$.

\item В конечном поле произведение всех ненулевых элементов не равно единице. Докажите, что сумма всех элементов поля равна нулю.

\item В эллипсе с фокусами $F_1$ и $F_2$ проведена хорда $MN$, которая проходит через фокус $F_2$. На прямой $F_1F_2$ выбраны две точки $S$ и $T$ такие, что прямые $SM$ и $TN$ являются касательными к эллипсу. Точка $D$ симметрична $F_2$ относительно прямой $SM$, точка $E$ симметрична $F_2$ относительно $NT$. Прямые $DS$, $TE$ и $MN$ при пересечении образуют треугольник $ABC$ (точка $C$ не лежит на $MN$). Докажите, что:
\begin{itemize}
\item[а)] $CF_2$ --- медиана треугольника $ABC$;
\item[б)] $CF_1$ --- биссектриса треугольника $ABC$.
\end{itemize}

\item Максималист и минималист по очереди вписывают по одному числу в таблицу размера $n \times n$ (последовательно, строчка за строчкой, слева направо и сверху вниз). Каким окажется ранг получившейся матрицы, если максималист изо всех сил старается его максимизировать, а минималист --- минимизировать? (Ответ может зависеть от $n$ и от того, кто делает первый ход.) 
 
\item Функция $f : (1, +\infty) \rightarrow \mathbb{R}$ дифференцируема на всей области определения. Известно, что  $$\displaystyle f'(x) = f\left( \frac{x}{x-1} \right) + f(x)$$ для всех $x > 1$ и $\displaystyle \lim\limits_{x \to \infty} \frac{f'(x)}{e^x} = 2$. Докажите, что $f(2) < 20,\hspace{-2pt}16$.

\end{enumerate}


\result
\begin{center}
\begin{tabular}{|l|l|l|c|c|c|c|c|c|c|c|c|}
\hline
№ & Участник & ВУЗ & Курс & 1 & 2 & 3 & 4 & 5 & 6 & $\Sigma$ & Диплом \\
\hline
1 & Журавская Александра & КФ МГУ & 2 & 10 & 9 & 0 & 0 & 9 & 0 & 28 & 1 степени \\
\hline
2 & Аманкелды Акежан & НУ & 3 & 3 & 9 & 2 & 0 & 1 & 7 & 22 & 2 степени \\
\hline
3 & Турганбаев Сатбек & КФ МГУ & 2 & 10 & 7 & 3 & 0 & 0 & 1 & 21 & 2 степени \\
\hline
4 & Полищук Руслан & НУ & 1 & 10 & 7 & 0 & 0 & 0 & 3 & 20 & 3 степени \\
\hline
5 & Жанахметов Султан & НУ & 2 & 10 & 5 & 0 & 0 & 1 & 3 & 19 & 3 степени \\
\hline
6 & Токтаганов Адиль & КФ МГУ & 2 & 8 & 0 & 0 & 10 & 1 & 0 & 19 & 3 степени \\
\hline
\end{tabular}
\end{center}

\newpage


\head{IX олимпиада}{Казахстанский филиала МГУ имени М.В.Ломоносова}{Астана}{13 апреля}{2017}
\begin{enumerate}

\item Пусть $(T_n)_{n=1}^{\infty}$ --- последовательность натуральных чисел, заданная рекуррентно: $T_1 = T_2 = T_3 = 1$ и $T_{n+3} = T_{n+2} + T_{n+1} + T_n$ при $n \geqslant 1$. Вычислите сумму ряда $$\displaystyle \sum\limits_{n=1}^{\infty} \frac{T_n}{2^n},$$
если известно, что данный ряд сходится.

\item Найдите все простые $p$, запись которых в $k$-ичной системе счисления при некотором натуральном $k > 1$ содержит ровно $k$ различных цифр (старшая цифра не может быть нулём).

\item Докажите, что в любой группе квадрат произведения двух элементов порядка два и куб произведения двух элементов порядка три всегда являются коммутаторами.

\item Точка $P$ лежит внутри выпуклой области, ограниченной параболой $y = x^2$, но не лежит на оси $OY$. Обозначим через $S(P)$ множество всех точек, полученных отражением $P$ относительно всех касательных к параболе.

а) Докажите, что значение суммы 
$$\displaystyle\max_{(x, y) \in S(P)} y ~ + \min_{(x, y) \in S(P)} y$$ не зависит от выбора точки $P$.

б) Найдите геометрическое место точек $P$ таких, что $\displaystyle\max_{(x, y) \in S(P)} y = 0.$

\item Для каждой функции $f: [0, 1] \to \mathbb{R}$ обозначим через $s_n(f)$ и $S_n(f)$ нижнюю и верхнюю суммы Дарбу для функции $f$, соответствующие равномерному разбиению $[0, 1]$ на $n$ частей. Существует ли такая интегрируемая функция $f$, что $\displaystyle \sum_{n=1}^{\infty} s_n(f)$ сходится, а
$\displaystyle\sum_{n=1}^{\infty} S_n(f)$ расходится?

\item Некоторые участники математической олимпиады списали решения некоторых задач у своих товарищей. Докажите, что можно с позором выгнать часть участников так, чтобы получилось, что более четверти от общего числа списанных решений было списано выгнанными участниками у не выгнанных.

\end{enumerate}


\result
\begin{center}
\begin{tabular}{|l|l|l|c|c|c|c|c|c|c|c|c|}
\hline
№ & Участник & ВУЗ & Курс & 1 & 2 & 3 & 4 & 5 & 6 & $\Sigma$ & Диплом \\
\hline
1 & Жанбырбаев Есеналы & КБТУ & 2 & 10 & 10 & 10 & 3 & 0 & 10 & 43 & 1 степени \\
\hline
2 & Бекмаганбетов Бекарыс & КФ МГУ & 1 & 10 & 10 & 10 & 2 & 10 & 0 & 42 & 2 степени \\
\hline
3 & Сайланбаев Алибек & НУ & 4 & 10 & 10 & 10 & 2 & 0 & 8 & 40 & 2 степени \\
\hline
4 & Аманкелды Акежан & НУ & 4 & 9 & 9 & 10 & 5 & 0 & 0 & 33 & 3 степени \\
\hline
5 & Жанахметов Султан & НУ & 3 & 10 & 10 & 10 & 2 & 0 & 0 & 32 & 3 степени \\
\hline
6 & Шакиев Александр & МУИТ & 2 & 10 & 9 & 10 & 0 & 0 & 0 & 29 & 3 степени \\
\hline
\end{tabular}
\end{center}

\newpage


\head{X олимпиада}{Казахстанско-Британский технический университет}{Алматы}{3 апреля}{2018}
\begin{enumerate}

\item Последовательность многочленов $P_n$ равномерно сходится на всей оси $f: \mathbb{R} \to \mathbb{R}$. Докажите, что $f$ является многочленом.

\item Докажите, что если непрерывно дифференцируемая функция $f: \mathbb{R} \to \mathbb{R}$ удовлевторяет тождеству $f(x) = \alpha f(x / 2)$, а $|\alpha| < 2$, то при $|\alpha| = 1$ функция $f$ --- произвольная постоянная, а при остальных $\alpha$ --- нулевая.

\item Две непрерывно дифференцируемые на $[0; a]$ функции $f_0$, $f_1$ принимают неположительные значения и $f_0(0) = f_1(0) = 0$. Докажите, что если при всех $x$ выполняется неравенство $f'_0(x) + x f'_1(x) \geqslant 0$ на отрезке $[0; a]$, то обе функции являются тождественно нулевыми.

\item Известно, что на графике многочлена $P$ можно отметить $n$ точек, являющихся вершинами правильного $n$-угольника. Доказать, что его степень не меньше $n - 1$.

\item Если три вектора $(u_1, u_2, u_3)$, $(v_1, v_2, v_3)$, $(w_1, w_2, w_3)$ с ненулевыми координатами попарно ортогональны, то векторы $\left(\frac{1}{u_1}, \frac{1}{u_2}, \frac{1}{u_3} \right)$, $\left(\frac{1}{v_1}, \frac{1}{v_2}, \frac{1}{v_3} \right)$, $\left(\frac{1}{w_1}, \frac{1}{w_2}, \frac{1}{w_3} \right)$ с обратными координатами компланарны, т.е. лежат в одной плоскости.
\end{enumerate}


\result
\begin{center}
\begin{tabular}{|l|l|l|c|c|c|c|c|c|c|c|}
\hline
№ & Участник & ВУЗ & Курс & 1 & 2 & 3 & 4 & 5 & $\Sigma$ & Диплом \\
\hline
1 & Бекмаганбетов Бекарыс & КФ МГУ & 2 & 10 & 8 & 10 & 8 & 10 & 46 & 1 степени \\
\hline
2 & Жанбырбаев Есеналы & КБТУ & 2 & 0 & 10 & 10 & 8 & 10 & 38 & 2 степени \\
\hline
3 & Шакиев Александр & МУИТ & 2 & 0 & 10 & 0 & 8 & 10 & 28 & 2 степени \\
\hline
4 & Куанышабай Максат & НУ & 2 & 0 & 4 & 10 & 1 & 10 & 25 & 3 степени \\
\hline
5 & Жакатаев Еркебулан & КазНУ & 4 & 10 & 10 & 0 & 2 & 0 & 22 & 3 степени \\
\hline
6 & Аскергалиев Ануар & КФ МГУ & 2 & 0 & 3 & 0 & 5 & 10 & 18 & 3 степени \\
\hline
\end{tabular}
\end{center}

\newpage

\addcontentsline{toc}{section}{Республиканская студенческая олимпиада по математическому и компьютерному моделированию}

\head{VI олимпиада}{Казахский национальный университет имени аль-Фараби}{Алматы}{27 марта}{2014}
\begin{flushright}
Стоимость задач: \\
10, 20, 25, 15 и 20 баллов соответственно.\\
\end{flushright}

\begin{enumerate}
\item Исследовать на сходимость ряд $\sum\limits_{n=1}^{\infty} a_n$, общий член которого имеет вид 
$$a_n = \left( \frac{1 \cdot 3 \cdot 5 \cdot \ldots \cdot (2n - 1)}{2 \cdot 4 \cdot 6 \cdot \ldots \cdot 2n} \right)^2.$$

\item Рассмотрим обыкновенное дифференциальное уравнение Клеро $y = p \cdot x + f(p)$, где $p = y'$ и $f$ --- дифференцируемая функция. Доказать, что если $f'$ --- монотонная функция, то особое решение уравнения Клеро имеет ровно одну общую точку с любым частным решением.

\item Во время недавних раскопок на Марсе были обнаружены листы бумаги с таинственными символами на них. После долгих исследований учёные пришли к выводу, что надписи на них на самом деле могли быть обычными числовыми равенствами. Кроме того, из других источников было получено веское доказательство того, что марсиане знали только три операции: сложение, умножение и вычитание (марсиане никогда не использовали <<унарный минус>>: вместо <<$-5$>>они писали <<$0-5$>>). Также ученые доказали, что марсиане не наделяли операции разным приоритетом, а просто вычисляли выражения (если в них не было скобок) слева направо: например, $3+3*5$ у них равнялось 30, а не 18. К сожалению, символы арифметических действий стерлись. Например, если была запись <<$18=7 (5 3) 2$>>, то возможно восстановить эту запись как <<$18=7+(5-3)*2$>>. Требуется написать программу, находящую требуемую расстановку знаков или сообщающую, что таковой не существует.

Первая строка входного файла {\tt INPUT.TXT} состоит из натурального числа, не превосходящего $2^{30}$, знака равенства, и последовательности натуральных чисел (не более десяти), произведение которых также не превосходит $2^{30}$. Некоторые группы чисел (одно или более) могут быть окружены скобками. Длина входной строки не будет превосходить 80 символов, и других ограничений на количество и вложенность скобок нет. Между двумя соседними числами, не разделенными скобками, всегда будет хотя бы один пробел, во всех остальных местах может быть любое (в том числе и 0) число пробелов (естественно, внутри числа пробелов нет).

В выходной файл {\tt OUTPUT.TXT} необходимо вывести одну строку, содержащую полученное равенство (т.е., исходное равенство со вставленными знаками арифметических действий без лишних пробелов). В случае если требуемая расстановка знаков невозможна, вывести строку, состоящую из единственного числа <<$-1$>>. Выходная строка не должна содержать пробелов.

\item Вычислите определитель

$$
\left| \begin{array}{cccc}
P(x) & P(x+1) & \dots & P(x+2014) \\
P'(x) & P'(x+1) & \dots & P'(x+2014) \\
\dots & \dots & \dots & \dots \\
P^{(2014)}(x) & P^{(2014)}(x+1) & \dots & P^{(2014)}(x+2014) \\
\end{array}\right|,
$$
где $P(x) = x (x+1) \cdot \ldots \cdot (x + 2014)$.

\item Внутри тетраэдра $ABCD$ расположена точка $O$ так, что прямые $AO$, $BO$, $CO$, $DO$ пересекают грани $BCD$, $ACD$, $ABD$, $ABC$ тетраэдра в точках $A_1$, $B_1$, $C_1$, $D_1$ соответственно. Причем отношения 
$$
\frac{AO}{A_1O}, 
\frac{BO}{B_1O},
\frac{CO}{C_1O},
\frac{DO}{D_1O}
$$
равны одному и тому же числу. Найти все значения, которые может принимать это число.
\end{enumerate}

\result
\begin{center}
\begin{tabular}{|l|l|l|c|c|c|c|c|c|c|c|}
\hline
№ & Участник & ВУЗ & Курс & 1 & 2 & 3 & 4 & 5  & Итог & Диплом \\
\hline
1 & Даржанова Асель & КазНУ & 3 & 10 & 17 & 0 & 10 & 15 & 52 & 1 степени \\
\hline
2 & Каныбек Тогжан & КазНУ & 3 & 10 & 15 & 0 & 0 & 20 & 45 & 2 степени \\
\hline
3 & Амир Мирас & КФ МГУ & 1 & 3 & 0 & 0 & 15 & 20 & 38 & 2 степени \\
\hline
4 & Тлеуова Гайни & КазНУ & 3 & 0 & 17 & 0 & 0 & 20 & 37 & 3 степени \\
\hline
5 & Бисенгалиева Асем & КазНУ & 3 & 0 & 17 & 0 & 0 & 15 & 32 & 3 степени \\
\hline
6 & Таскынов Ануар & КФ МГУ & 1 & 0 & 0 & 0 & 3 & 18 & 21 & 3 степени \\
\hline
\end{tabular}
\end{center}


\newpage

\head{VII олимпиада}{Казахский национальный университет имени аль-Фараби}{Алматы}{26 марта}{2015}
\begin{enumerate}
\item а) Вычислить предел:
$$\lim_{n \to \infty} \sin^2 \left( \pi \sqrt{n^2 + 2n}\right).$$

б) Зная, что $\int\limits_{0}^{1} 
\frac{\ln(1+x)}{x} dx = \frac{\pi^2}{12}$, вычислить
$$\int\limits_{0}^{1} 
\frac{\ln(1-x^3)}{x} dx.$$

\item Числа 53295, 67507, 88825, 81719, 39083 делятся на 3553. Не вычисляя определитель матрицы $$A = 
\begin{pmatrix}
5 & 3 & 2 & 9 & 5 \\
6 & 7 & 5 & 0 & 7 \\
8 & 8 & 8 & 2 & 5 \\
8 & 1 & 7 & 1 & 9 \\
3 & 9 & 0 & 8 & 3 \\
\end{pmatrix},
$$
доказать, что он делится на 3553.

\item Доказать тождество:
$$\frac{x}{1} + 
\frac{x^3}{1 \cdot 3 }+
\frac{x^5}{1 \cdot 3 \cdot 5} + \ldots = 
\exp^{x^2/2} \cdot \int_{0}^{x} \exp^{-\frac{t^2}{2}} dt.$$

\item Даны две строки, представляющие числа $A$ и $B$ в фибоначчиевой системе счисления. Описать программу, которая находит строку, представляющую число $A + B$ в фибоначчиевой системе счисления. Числа $A$ и $B$ могут превышать максимальное допустимое значение в стандартных целочисленных типах.

Примечание: Числа Фибоначии $F_1$, $F_2$, $\ldots$ определяются начальными значениями:
$$F_1 = 1, F_2 = 2, F_{N+1} = F_N + F_{N-1}.$$
Рассмотрим фибоначчиеву систему счисления с двумя цифрами 0 и 1, в которой, в отличие от двоичной системы весами являются не степени двойки 1, 2, 4, 8, 16, \ldots, а числа Фибоначчи 1, 2, 3, 5, 8, 13, \ldots. В этой системе счисления каждое положительное целое число единственным образом представляется в виде строки нулей и единиц, которая начинается с 1 и в которой нет двух единиц, стоящих рядом.

\item Совсем недавно Вася занялся программированием и решил реализовать собственную программу для игры в шахматы. Но у него возникла проблема определения правильности хода конем, который делает пользователь. Т.е. если пользователь вводит значение <<C7-D5>>, то программа должна определить это как правильный ход, если же введено <<E2-E4>>, то ход неверный. Так же нужно проверить корректность записи ввода: если например, введено <<D9-N5>>, то программа должна определить данную запись как ошибочную. Помогите ему осуществить эту проверку!

Входные данные:

В единственной строке входного файла {\tt INPUT.TXT} записан текст хода (непустая строка), который указал пользователь. Пользователь не может ввести строку, длиннее 5 символов.

Выходные данные:

В выходной файл {\tt OUTPUT.TXT} нужно вывести <<YES>>, если указанный ход конем верный, если же запись корректна (в смысле правильности записи координат), но ход невозможен, то нужно вывести <<NO>>. Если же координаты не определены или заданы некорректно, то вывести сообщение <<ERROR>>.
\end{enumerate}

\result
\begin{center}
\begin{tabular}{|l|l|l|c|c|}
\hline
№ & Участник & ВУЗ & Курс & Диплом \\
\hline
1 & Амир Мирас & КФ МГУ & 2 & 1 степени \\
\hline
2 & Жусупов Али & КФ МГУ & 1 & 2 степени \\
\hline
3 & Мусакулова Гульдерайым & КазНУ & 4 & 2 степени \\
\hline
4 & Ахмер Ермек & КазНУ & 4 & 3 степени \\
\hline
5 & Каныбек Тогжан & КазНУ & 4 & 3 степени \\
\hline
6 & Иманбекова Жанна & КазНУ & ? & 3 степени \\
\hline
\end{tabular}
\end{center}


\newpage


\head{VIII олимпиада}{Казахстанский филиала МГУ имени М.В.Ломоносова}{Астана}{1 апреля}{2016}
\begin{flushright}
Стоимость задач: \\
10 баллов каждая задача.\\
\end{flushright}

\begin{enumerate}
\item Введём функцию
$$
f(n) = \bigl[\sqrt 1\,\bigr] + \bigl[\sqrt 2\,\bigr] + \bigl[\sqrt 3\,\bigr] + \hdots + \bigl[\sqrt{n^2-1}\,\bigr] + \bigl[\sqrt{n^2}\,\bigr],
$$
где $[x]$ --- наибольшее целое число, не превышающее $x$. Опишите функцию, которая вычисляет $f(n)$ для данного натурального $n$, не используя при этом операцию извлечения корня и вещественную арифметику.

\item На декартовой координатной плоскости нарисованы две полупараболы: график функции $y = x^2$ $(x \geqslant 0)$ и его копия, повёрнутая на прямой угол по часовой стрелке. Эти две кривые отсекают от прямой, параллельной оси ординат, отрезок длины $L$. Обозначим через $S(L)$ --- площадь отсечённой фигуры.

a) Докажите, что $S(L) > 1$ при $L > 2$;

б) Напишите функцию, которая вычисляет $S(L)$ для данного положительного вещественного числа $L$.



\item Найдите все дифференцируемые функции $f\colon \mathbb R\rightarrow\mathbb R$, удовлетворяющие соотношению
$$
f(x-y) + f(x+y) = f'(x^2 + y^2)
$$
для любых $x,y\in\mathbb R$.



\item  Дана функция $f\colon [0, 2n]\rightarrow\mathbb R$. Пусть $f_i = f(i)$ --- значения функции во всех целых $i$ от 0 до $2n$. Дана переменная $S$ вещественного типа с начальным значением 0. За один ход робот может выбрать целое $i$ от 1 до $2n-1$, затем добавить к переменной $S$ или вычесть из нее среднее арифметическое значений функции $f(x)$ в узлах~$i-1$,~$i$,~$i+1$:
$$S := S \pm \frac{f_{i-1}+f_i+f_{i+1}}{3}.$$ 
Может ли робот за конечное число ходов получить в переменной $S$ значение
$$I = \frac{1}{3} \left(f_0 + 4\sum\limits_{k=1}^{n}{f_{2k-1}} + 2\sum\limits_{k=1}^{n-1}{f_{2k}} + f_{2n}\right),$$
которое является приближением интеграла $\displaystyle \int\limits_0^{2n} f(x)\,dx$, если\\
а) $f(0) = f(2n) = 0$;\\
б) $f(0) \ne 0$, $f(2n) \ne 0$?

\item  Дана некоторая условная машина, состоящая из памяти в $n$ бит и указателя, который в каждый отдельный момент находится над какой-то из этих n ячеек. Перед запуском программы в память записывается некоторое натуральное число $m$ в двоичной системе счисления, а указатель устанавливается над крайним правым (младшим) битом числа. Язык программирования для этой машины состоит из следующих команд:

\begin{center}
\begin{tabular}{|c|c|p{14cm}|}
\hline
{\bf L} & left & сместить указатель налево на одну ячейку, если это возможно, иначе завершить программу\\
\hline
{\bf R} & right & сместить указатель направо на одну ячейку, если это возможно, иначе завершить программу\\
\hline
{\bf C} & change & изменить значение бита в текущей ячейке на противоположное\\
\hline
{\bf A} & again & перейти к выполнению первой команды\\
\hline
{\bf S} & skip & пропустить две следующие команды, если в текущей ячейке 0\\
\hline
{\bf F} & finish & завершить выполнение программы\\
\hline
\end{tabular}
\end{center}

Команды записываются в одну строку и выполняются в последовательном порядке, слева направо. При этом запись программы обязана оканчиваться командой \textbf{A} или \textbf{F}. Напишите для этой абстрактной машины следующие программы:

а) заменить данное число на $(m - 1)$;

б) заменить данное число на $(2^n - m - 1)$;

в) изменить на противоположный его старший (крайний слева) бит.

\textit{Примеры:}

а) программа, обнуляющая все ячейки: \textbf{SSCLA};

б) программа, которая изменяет второй справа бит, если крайний справа бит нулевой: \textbf{SFFLCF}.


\item Из квадратной однородной пластины со стороной 1 случайным образом вырезается квадрат со сторонами, равными $2a$ и параллельными сторонам исходного квадрата. При этом центр квадрата --- это случайная величина, равномерно распределённая по всем допустимым положениям (квадрат со стороной $(1 - 2a)$).

a) Найдите вероятность $p(a)$ того, что центр тяжести полученной фигуры лежит в вырезанной области.

б) Опишите функцию $p(a)$, которая вычисляет указанную вероятность приблизительно, считая при этом, что нам не известен метод нахождения центра тяжести произвольной фигуры, однако мы можем найти центр тяжести конечного множества точек одинаковой массы.

\end{enumerate}


\result
\begin{center}
\begin{tabular}{|l|l|l|c|c|c|c|c|c|c|c|c|}
\hline
№ & Участник & ВУЗ & Курс & 1 & 2 & 3 & 4 & 5 & 6 & $\Sigma$ & Диплом \\
\hline
1 & Батырхан Орынкул & НУ & 3 & 10 & 10 & 1 & 5 & 9 & 6 & 41 & 1 степени \\
\hline
2 & Абайулы Ерулан & КФ МГУ & 2 & 5 & 10 & 2 & 2 & 10 & 0 & 29 & 2 степени \\
\hline
3 & Камалбеков Тимур & КФ МГУ & 2 & 8 & 7 & 1 & 0 & 9 & 2 & 27 & 2 степени \\
\hline
4 & Омаров Темирхан & КФ МГУ & 2 & 4 & 10 & 0 & 2 & 10 & 0 & 22 & 3 степени \\
\hline
5 & Сайланбаев Алибек & НУ & 3 & 7 & 5 & 3 & 1 & 6 & 0 & 22 & 3 степени \\
\hline
6 & Иманмалик Ержан & НУ & 2 & 9 & 3 & 3 & 5 & 1 & 0 & 21 & 3 степени \\
\hline
\end{tabular}
\end{center}


\newpage


\head{IX олимпиада}{Назарбаев Университет}{Астана}{13 апреля}{2017}
\begin{enumerate}
\item Назовем натуральное число $a$ <<хорошим>>, если его можно представить в виде $a = 10^e (1+m)$, где $e$ и $m$ такие целые числа, что $0 \leqslant e < 224$ и $0 \leqslant m < 10^{100}$.

a) Найдите минимальное натуральное число, которое не является <<хорошим>>;

б) Найдите количество <<хороших>> натуральных чисел.

\item Сваха собрала базу данных из $n$ мужчин и $n$ женщин и желает их всех
переженить. Проблема осложняется тем, что не все пары <<мужчина --- женщина>> психологически совместимы (то есть, поженить такую пару нельзя). Назовем любовным циклов список из $k > 2$ различных персон $X_1$, $\dots$, $X_k$, таких, что $X_1$ совместим с $X_2$, $X_2$ с $X_3$, $\dots$, $X_{k-1}$ с $X_k$, $X_k$ с $X_1$. Собрав всю информацию о взаимной совместимости, сваха обнаружила, что в ее базе всего лишь один любовный цикл. Обозначим $N$ общее количество способов которыми сваха сможет поженить всех. Два способа считаются различными, если хотя бы один человек в них вступает в брак с разными партнерами. Какие значения может принимать $N$?

\item В тридевятом царстве в тридесятом государстве жил был король-самодур и математик при его дворе. В государстве было $n$ городов и длины дорог между ними были известны (они принимали натуральные значения и были не длиннее 100 км). Король попросил математика найти длину наикратчайшего пути между западной и восточной столицами с допустимой ошибкой не более 1\%, пригрозив при этом казнью за любое сравнение чисел (операции $\min{x, y}$ и $\max{x, y}$ тоже запрещены). Разрешены лишь 
\begin{itemize}
\item суммирование;
\item умножение;
\item деление;
\item возведение в степень;
\item вычисление логарифма.
\end{itemize}
Как быть математику? Опишите его алгоритм (как можно более оптимальный по вычислительным затратам). Оцените вычислительную сложность этого алгоритма (то есть, получите верхнюю оценку $g(n)$ на общее количество вышеперечисленных операций в данном алгоритме).

\item Вам дали задание написать программу которая должна делать следующее:
\begin{itemize}
\item вначале прочитать файл с n вещественными числами $a_1$, $a_2$, $\dots$, $a_n$ и создать массив $A[i] = a_i$, $1 \leqslant i \leqslant n$.
\item если вы подаете ей на вход пару $(i, j)$ такую, что $1 \leqslant i \leqslant j \leqslant n$, то
\end{itemize}
программа должна возвратить $\max_{i \leqslant k \leqslant j} a_k$. Ваша программа, таким образом, существует в двух режимах:
\begin{enumerate}
\item в режиме препроцессинга массива $A$ --- в этом режиме вы можете обработать массив $A$, создать какие-то структуры данных, чтобы более эффективно выполнить работу на втором этапе; 
\item в режиме расчета для заданной пары $(i, j)$ ответа $\max_{i \leqslant k \leqslant j} a_k$.
\end{enumerate}
Опишите алгоритм, в котором
\begin{itemize}
\item Первый этап занимает $O(n^2)$ арифметических операций;
\item Второй этап занимает $O(1)$ арифметических операций;
\end{itemize}
Опишите алгоритм, в котором
\begin{itemize}
\item Первый этап занимает $O(n \log n)$ арифметических операций
\item Второй этап занимает $O(1)$ арифметических операций
\end{itemize}

\end{enumerate}


\result
\begin{center}
\begin{tabular}{|l|l|l|c|c|}
\hline
№ & Участник & ВУЗ & Курс & Диплом \\
\hline
1 & Орынкул Батырхан & НУ & 4 & 1 степени \\
\hline
2 & Калмурзаев Сергазы & НУ & 3 & 2 степени \\
\hline
3 & Касенов Бекжан & НУ & 4 & 2 степени \\
\hline
4 & Жусупов Али & ЕНУ & 2 & 3 степени \\
\hline
5 & Шакиев Александр & МУИТ & 1 & 3 степени \\
\hline
6 & ? & КазНУ & ? & 3 степени \\
\hline
\end{tabular}
\end{center}


\newpage


\head{X олимпиада}{Назарбаев Университет}{Астана}{20 апреля}{2018}
\begin{flushright}
Стоимость задач: \\
7 баллов каждая задача.\\
\end{flushright}

\begin{enumerate}
\item Мяч весом $0,1$ килограмма подбрасывают вверх с земли с начальной скоростью $20$ м/с. Сила сопротивления воздуха величиной $|v|^3/1020,4$ направлена в сторону противоположную скорости мяча, где скорость v измеряется в м/с. Найдите формулу для вычисления времени, которое требуется мячу для достижения максимальной высоты над уровнем земли. 

\item Допустим $1 < p < 2$ и дана функция $\varphi(x) = |x|^{p-2} x$, где $x \in \mathbb{R}$. Докажите неравенство
$$|\varphi(x) - \varphi(y)| \leqslant 2^{2-p} |x-y|^{p-1}.$$
Также покажите, что равенство имеет место если $y = -x$.

\item Вычислите точное значение $\cos\left(\frac{\pi}{10}\right)$.

\item Вычислите интеграл
$$\int_{0}^{1} \sqrt{-\ln(x)} dx.$$

\item Даны 52 точки $P_1(x_1, y_1)$, $P_2(x_2, y_2)$, \ldots, $P_{52}(x_{52}, y_{52})$, расположенные в квадрате, длина стороны которого равна 7. Покажите, что среди этих 52 точек всегда можно найти 3 точки,
которые расположены внутри круга радиуса 1.
\end{enumerate}

\result
\begin{center}
\begin{tabular}{|l|l|l|c|c|c|c|c|c|c|c|}
\hline
№ & Участник & ВУЗ & Курс & 1 & 2 & 3 & 4 & 5  & $\Sigma$ & Диплом \\
\hline
1 & Бекмаганбетов Бекарыс & КФ МГУ & 2 & 7 & 7 & 7 & 7 & 7 & 35 & 1 степени \\
\hline
2 & Аскергалиев Ануар & КФ МГУ & 2 & 4 & 0 & 7 & 6 & 2 & 19 & 2 степени \\
\hline
3 & Шарипов Азат & КФ МГУ & 2 & 2 & 0 & 0 & 5 & 1.5 & 8.5 & 2 степени \\
\hline
4 & Сабирова Роза & КазНУ & ? & 3 & 0 & 0 & 2 & 3 & 8 & 3 степени \\
\hline
5 & Абай Азат & КазНУ & ? & 4 & 1 & 0 & 0 & 1 & 6 & 3 степени \\
\hline
6 & Ногаева Аида & КазНУ & 3 & 3 & 1 & 0 & 0 & 1 & 5 & 3 степени \\
\hline
\end{tabular}
\end{center}


\newpage
\end{document}