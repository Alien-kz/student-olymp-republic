\documentclass[12pt,a4paper]{article}
\usepackage[utf8]{inputenc}
\usepackage[T2A]{fontenc}
\usepackage[russian]{babel}
\usepackage{amsmath}
\usepackage{amsfonts}
\usepackage{amssymb}
\usepackage[left=1.5cm,right=1.5cm,top=2cm,bottom=2cm]{geometry}

\newcommand{\head}[5]{ %hoster city date
	\addcontentsline{toc}{subsection}{#1 (#5)}
	\begin{center}
	#1 \\
	#2 \\
	#3 \\
	#4 #5\\
	\end{center}
}
\newcommand{\result}{
	\begin{center}
	\textbf{Результаты}
	\end{center}
}

\usepackage[explicit]{titlesec}
\titleformat{\section}{\normalfont\bfseries\center}{}{1em}{#1}
\titleformat{\subsection}{\normalfont\bfseries\center}{}{1em}{#1}


\renewpagestyle{headings}{
\sethead[][][]
{}{\subsectiontitle}{}
\headrule
\setfoot[\thepage][][]
{}{}{\thepage}
}

\begin{document}

\newpage

\pagestyle{headings}

\addcontentsline{toc}{section}{О республиканской олимпиаде}

Республиканская студенческая предметная олимпиада проводится ежегодно весной по поручению министерства образования и науки Республики Казахстан. Олимпиада по каждой специальности проводится отдельно. В данной брошюре собраны олимпиады по двум специальностям: <<Математика>>, <<Математическое и компьютерное моделирование>>. Впервые такая олимпиада прошла в 2009 году на базе Казахского национального университета имени аль-Фараби (г. Алматы). Вплоть до 2015 года олимпиада не меняла свое место проведения по этим направлениям. В 2016 году олимпиады по данным направлениям прошли в Казахстанского филиале МГУ имени М.~В.~Ломомносова (г. Астана), что стало традицией передавать право проведения олимпиады университету, представитель которого стал победителем в личном зачете.


Задачи на олимпиадах, как правило, доступны хорошо подготовленным студентам 2-4 курса. Желающим углубить свои навыки в решении студенческих олимпиадных задач рекомендуется список литературы:

\begin{enumerate}
\item Садовничий В.А., Подколзин А.С. Задачи студенческих олимпиад по математике. М.: Наука. 1978. --- 208 с.
\item Садовничий В.А., Григорьян А.А., Конягин С.В. Задачи студенческих математических олимпиад. Издательство МГУ, 1987. --- 309 с.
\item Gelca R., Andreescu T. Putnam and Beyond. 2007. --- 850 pp.
\item Меньшиков Ф. Олимпиадные задачи по программированию. 2007. --- 320 с.
\end{enumerate}


\newpage

\tableofcontents

\newpage

\addcontentsline{toc}{section}{Республиканская студенческая олимпиада по математике}

\head{VI олимпиада}{Казахский национальный университет имени аль-Фараби}{Алматы}{27 марта}{2014}
\begin{center}
\begin{tabular}{|l|l|l|c|c|c|c|c|c|c|c|}
\hline
№ & Участник & ВУЗ & Курс & 1 & 2 & 3 & 4 & 5  & Итог & Диплом \\
\hline
1 & Даржанова Асель & КазНУ & 3 & 10 & 17 & 0 & 10 & 15 & 52 & 1 \\
\hline
2 & Каныбек Тогжан & КазНУ & 3 & 10 & 15 & 0 & 0 & 20 & 45 & 2 \\
\hline
3 & Амир Мирас & КФ МГУ & 1 & 3 & 0 & 0 & 15 & 20 & 38 & 2 \\
\hline
4 & Тлеуова Гайни & КазНУ & 3 & 0 & 17 & 0 & 0 & 20 & 37 & 3 \\
\hline
5 & Бисенгалиева Асем & КазНУ & 3 & 0 & 17 & 0 & 0 & 15 & 32 & 3 \\
\hline
6 & Таскынов Ануар & КФ МГУ & 1 & 0 & 0 & 0 & 3 & 18 & 21 & 3 \\
\hline
\end{tabular}
\end{center}


\result
\begin{center}
\begin{tabular}{|l|l|l|c|c|c|c|c|c|c|c|}
\hline
№ & Участник & ВУЗ & Курс & 1 & 2 & 3 & 4 & 5  & Итог & Диплом \\
\hline
1 & Даржанова Асель & КазНУ & 3 & 10 & 17 & 0 & 10 & 15 & 52 & 1 \\
\hline
2 & Каныбек Тогжан & КазНУ & 3 & 10 & 15 & 0 & 0 & 20 & 45 & 2 \\
\hline
3 & Амир Мирас & КФ МГУ & 1 & 3 & 0 & 0 & 15 & 20 & 38 & 2 \\
\hline
4 & Тлеуова Гайни & КазНУ & 3 & 0 & 17 & 0 & 0 & 20 & 37 & 3 \\
\hline
5 & Бисенгалиева Асем & КазНУ & 3 & 0 & 17 & 0 & 0 & 15 & 32 & 3 \\
\hline
6 & Таскынов Ануар & КФ МГУ & 1 & 0 & 0 & 0 & 3 & 18 & 21 & 3 \\
\hline
\end{tabular}
\end{center}

\newpage


\head{VII олимпиада}{Казахский национальный университет имени аль-Фараби}{Алматы}{26 марта}{2015}
\begin{center}
\begin{tabular}{|l|l|l|c|c|}
\hline
№ & Участник & ВУЗ & Курс & Диплом \\
\hline
1 & Амир Мирас & КФ МГУ & 2 & 1  \\
\hline
2 & Жусупов Али & КФ МГУ & 1 & 2  \\
\hline
3 & Мусакулова Гульдерайым & КазНУ & 4 & 2  \\
\hline
4 & Ахмер Ермек & КазНУ & 4 & 3  \\
\hline
5 & Каныбек Тогжан & КазНУ & 4 & 3  \\
\hline
6 & Иманбекова Жанна & КазНУ & ? & 3  \\
\hline
\end{tabular}
\end{center}


\result
\begin{center}
\begin{tabular}{|l|l|l|c|c|}
\hline
№ & Участник & ВУЗ & Курс & Диплом \\
\hline
1 & Амир Мирас & КФ МГУ & 2 & 1  \\
\hline
2 & Жусупов Али & КФ МГУ & 1 & 2  \\
\hline
3 & Мусакулова Гульдерайым & КазНУ & 4 & 2  \\
\hline
4 & Ахмер Ермек & КазНУ & 4 & 3  \\
\hline
5 & Каныбек Тогжан & КазНУ & 4 & 3  \\
\hline
6 & Иманбекова Жанна & КазНУ & ? & 3  \\
\hline
\end{tabular}
\end{center}


\newpage


\head{VIII олимпиада}{Казахстанский филиала МГУ имени М.В.Ломоносова}{Астана}{1 апреля}{2016}
\input{problems/math/2016.tex}

\result
\input{results/math/2016.tex}

\newpage


\head{IX олимпиада}{Казахстанский филиала МГУ имени М.В.Ломоносова}{Астана}{13 апреля}{2017}
\begin{center}
\begin{tabular}{|l|l|l|c|c|}
\hline
№ & Участник & ВУЗ & Курс & Диплом \\
\hline
1 & Орынкул Батырхан & НУ & 4 & 1 \\
\hline
2 & Калмурзаев Сергазы & НУ & 3 & 2 \\
\hline
3 & Касенов Бекжан & НУ & 4 & 2 \\
\hline
4 & Жусупов Али & ЕНУ & 2 & 3 \\
\hline
5 & Шакиев Александр & МУИТ & 1 & 3 \\
\hline
6 & ? & КазНУ & ? & 3 \\
\hline
\end{tabular}
\end{center}


\result
\begin{center}
\begin{tabular}{|l|l|l|c|c|}
\hline
№ & Участник & ВУЗ & Курс & Диплом \\
\hline
1 & Орынкул Батырхан & НУ & 4 & 1 \\
\hline
2 & Калмурзаев Сергазы & НУ & 3 & 2 \\
\hline
3 & Касенов Бекжан & НУ & 4 & 2 \\
\hline
4 & Жусупов Али & ЕНУ & 2 & 3 \\
\hline
5 & Шакиев Александр & МУИТ & 1 & 3 \\
\hline
6 & ? & КазНУ & ? & 3 \\
\hline
\end{tabular}
\end{center}


\newpage


\head{X олимпиада}{Казахстанско-Британский технический университет}{Алматы}{3 апреля}{2018}
\begin{flushright}
Стоимость задач: \\
7 баллов каждая задача.\\
\end{flushright}

\begin{enumerate}
\item Мяч весом $0,1$ килограмма подбрасывают вверх с земли с начальной скоростью $20$ м/с. Сила сопротивления воздуха величиной $|v|^3/1020,4$ направлена в сторону противоположную скорости мяча, где скорость v измеряется в м/с. Найдите формулу для вычисления времени, которое требуется мячу для достижения максимальной высоты над уровнем земли. 

\item Допустим $1 < p < 2$ и дана функция $\varphi(x) = |x|^{p-2} x$, где $x \in \mathbb{R}$. Докажите неравенство
$$|\varphi(x) - \varphi(y)| \leqslant 2^{2-p} |x-y|^{p-1}.$$
Также покажите, что равенство имеет место если $y = -x$.

\item Вычислите точное значение $\cos\left(\frac{\pi}{10}\right)$.

\item Вычислите интеграл
$$\int_{0}^{1} \sqrt{-\ln(x)} dx.$$

\item Даны 52 точки $P_1(x_1, y_1)$, $P_2(x_2, y_2)$, \ldots, $P_{52}(x_{52}, y_{52})$, расположенные в квадрате, длина стороны которого равна 7. Покажите, что среди этих 52 точек всегда можно найти 3 точки,
которые расположены внутри круга радиуса 1.
\end{enumerate}

\result
\begin{center}
\begin{tabular}{|l|l|l|c|c|c|c|c|c|c|c|}
\hline
№ & Участник & ВУЗ & Курс & 1 & 2 & 3 & 4 & 5  & $\Sigma$ & Диплом \\
\hline
1 & Бекмаганбетов Бекарыс & КФ МГУ & 2 & 7 & 7 & 7 & 7 & 7 & 35 & 1  \\
\hline
2 & Аскергалиев Ануар & КФ МГУ & 2 & 4 & 0 & 7 & 6 & 2 & 19 & 2  \\
\hline
3 & Шарипов Азат & КФ МГУ & 2 & 2 & 0 & 0 & 5 & 1.5 & 8.5 & 2  \\
\hline
4 & Сабирова Роза & КазНУ & ? & 3 & 0 & 0 & 2 & 3 & 8 & 3  \\
\hline
5 & Абай Азат & КазНУ & ? & 4 & 1 & 0 & 0 & 1 & 6 & 3  \\
\hline
6 & Ногаева Аида & КазНУ & 3 & 3 & 1 & 0 & 0 & 1 & 5 & 3  \\
\hline
\end{tabular}
\end{center}


\newpage

\addcontentsline{toc}{section}{Республиканская студенческая олимпиада по математическому и компьютерному моделированию}

\head{VI олимпиада}{Казахский национальный университет имени аль-Фараби}{Алматы}{27 марта}{2014}
\begin{center}
\begin{tabular}{|l|l|l|c|c|c|c|c|c|c|c|}
\hline
№ & Участник & ВУЗ & Курс & 1 & 2 & 3 & 4 & 5  & Итог & Диплом \\
\hline
1 & Даржанова Асель & КазНУ & 3 & 10 & 17 & 0 & 10 & 15 & 52 & 1 \\
\hline
2 & Каныбек Тогжан & КазНУ & 3 & 10 & 15 & 0 & 0 & 20 & 45 & 2 \\
\hline
3 & Амир Мирас & КФ МГУ & 1 & 3 & 0 & 0 & 15 & 20 & 38 & 2 \\
\hline
4 & Тлеуова Гайни & КазНУ & 3 & 0 & 17 & 0 & 0 & 20 & 37 & 3 \\
\hline
5 & Бисенгалиева Асем & КазНУ & 3 & 0 & 17 & 0 & 0 & 15 & 32 & 3 \\
\hline
6 & Таскынов Ануар & КФ МГУ & 1 & 0 & 0 & 0 & 3 & 18 & 21 & 3 \\
\hline
\end{tabular}
\end{center}


\result
\begin{center}
\begin{tabular}{|l|l|l|c|c|c|c|c|c|c|c|}
\hline
№ & Участник & ВУЗ & Курс & 1 & 2 & 3 & 4 & 5  & Итог & Диплом \\
\hline
1 & Даржанова Асель & КазНУ & 3 & 10 & 17 & 0 & 10 & 15 & 52 & 1 \\
\hline
2 & Каныбек Тогжан & КазНУ & 3 & 10 & 15 & 0 & 0 & 20 & 45 & 2 \\
\hline
3 & Амир Мирас & КФ МГУ & 1 & 3 & 0 & 0 & 15 & 20 & 38 & 2 \\
\hline
4 & Тлеуова Гайни & КазНУ & 3 & 0 & 17 & 0 & 0 & 20 & 37 & 3 \\
\hline
5 & Бисенгалиева Асем & КазНУ & 3 & 0 & 17 & 0 & 0 & 15 & 32 & 3 \\
\hline
6 & Таскынов Ануар & КФ МГУ & 1 & 0 & 0 & 0 & 3 & 18 & 21 & 3 \\
\hline
\end{tabular}
\end{center}


\newpage

\head{VII олимпиада}{Казахский национальный университет имени аль-Фараби}{Алматы}{26 марта}{2015}
\begin{center}
\begin{tabular}{|l|l|l|c|c|}
\hline
№ & Участник & ВУЗ & Курс & Диплом \\
\hline
1 & Амир Мирас & КФ МГУ & 2 & 1  \\
\hline
2 & Жусупов Али & КФ МГУ & 1 & 2  \\
\hline
3 & Мусакулова Гульдерайым & КазНУ & 4 & 2  \\
\hline
4 & Ахмер Ермек & КазНУ & 4 & 3  \\
\hline
5 & Каныбек Тогжан & КазНУ & 4 & 3  \\
\hline
6 & Иманбекова Жанна & КазНУ & ? & 3  \\
\hline
\end{tabular}
\end{center}


\result
\begin{center}
\begin{tabular}{|l|l|l|c|c|}
\hline
№ & Участник & ВУЗ & Курс & Диплом \\
\hline
1 & Амир Мирас & КФ МГУ & 2 & 1  \\
\hline
2 & Жусупов Али & КФ МГУ & 1 & 2  \\
\hline
3 & Мусакулова Гульдерайым & КазНУ & 4 & 2  \\
\hline
4 & Ахмер Ермек & КазНУ & 4 & 3  \\
\hline
5 & Каныбек Тогжан & КазНУ & 4 & 3  \\
\hline
6 & Иманбекова Жанна & КазНУ & ? & 3  \\
\hline
\end{tabular}
\end{center}


\newpage


\head{VIII олимпиада}{Казахстанский филиала МГУ имени М.В.Ломоносова}{Астана}{1 апреля}{2016}
\input{problems/mcm/2016.tex}

\result
\input{results/mcm/2016.tex}

\newpage


\head{IX олимпиада}{Назарбаев Университет}{Астана}{13 апреля}{2017}
\begin{center}
\begin{tabular}{|l|l|l|c|c|}
\hline
№ & Участник & ВУЗ & Курс & Диплом \\
\hline
1 & Орынкул Батырхан & НУ & 4 & 1 \\
\hline
2 & Калмурзаев Сергазы & НУ & 3 & 2 \\
\hline
3 & Касенов Бекжан & НУ & 4 & 2 \\
\hline
4 & Жусупов Али & ЕНУ & 2 & 3 \\
\hline
5 & Шакиев Александр & МУИТ & 1 & 3 \\
\hline
6 & ? & КазНУ & ? & 3 \\
\hline
\end{tabular}
\end{center}


\result
\begin{center}
\begin{tabular}{|l|l|l|c|c|}
\hline
№ & Участник & ВУЗ & Курс & Диплом \\
\hline
1 & Орынкул Батырхан & НУ & 4 & 1 \\
\hline
2 & Калмурзаев Сергазы & НУ & 3 & 2 \\
\hline
3 & Касенов Бекжан & НУ & 4 & 2 \\
\hline
4 & Жусупов Али & ЕНУ & 2 & 3 \\
\hline
5 & Шакиев Александр & МУИТ & 1 & 3 \\
\hline
6 & ? & КазНУ & ? & 3 \\
\hline
\end{tabular}
\end{center}


\newpage


\head{X олимпиада}{Назарбаев Университет}{Астана}{20 апреля}{2018}
\begin{flushright}
Стоимость задач: \\
7 баллов каждая задача.\\
\end{flushright}

\begin{enumerate}
\item Мяч весом $0,1$ килограмма подбрасывают вверх с земли с начальной скоростью $20$ м/с. Сила сопротивления воздуха величиной $|v|^3/1020,4$ направлена в сторону противоположную скорости мяча, где скорость v измеряется в м/с. Найдите формулу для вычисления времени, которое требуется мячу для достижения максимальной высоты над уровнем земли. 

\item Допустим $1 < p < 2$ и дана функция $\varphi(x) = |x|^{p-2} x$, где $x \in \mathbb{R}$. Докажите неравенство
$$|\varphi(x) - \varphi(y)| \leqslant 2^{2-p} |x-y|^{p-1}.$$
Также покажите, что равенство имеет место если $y = -x$.

\item Вычислите точное значение $\cos\left(\frac{\pi}{10}\right)$.

\item Вычислите интеграл
$$\int_{0}^{1} \sqrt{-\ln(x)} dx.$$

\item Даны 52 точки $P_1(x_1, y_1)$, $P_2(x_2, y_2)$, \ldots, $P_{52}(x_{52}, y_{52})$, расположенные в квадрате, длина стороны которого равна 7. Покажите, что среди этих 52 точек всегда можно найти 3 точки,
которые расположены внутри круга радиуса 1.
\end{enumerate}

\result
\begin{center}
\begin{tabular}{|l|l|l|c|c|c|c|c|c|c|c|}
\hline
№ & Участник & ВУЗ & Курс & 1 & 2 & 3 & 4 & 5  & $\Sigma$ & Диплом \\
\hline
1 & Бекмаганбетов Бекарыс & КФ МГУ & 2 & 7 & 7 & 7 & 7 & 7 & 35 & 1  \\
\hline
2 & Аскергалиев Ануар & КФ МГУ & 2 & 4 & 0 & 7 & 6 & 2 & 19 & 2  \\
\hline
3 & Шарипов Азат & КФ МГУ & 2 & 2 & 0 & 0 & 5 & 1.5 & 8.5 & 2  \\
\hline
4 & Сабирова Роза & КазНУ & ? & 3 & 0 & 0 & 2 & 3 & 8 & 3  \\
\hline
5 & Абай Азат & КазНУ & ? & 4 & 1 & 0 & 0 & 1 & 6 & 3  \\
\hline
6 & Ногаева Аида & КазНУ & 3 & 3 & 1 & 0 & 0 & 1 & 5 & 3  \\
\hline
\end{tabular}
\end{center}


\newpage
\end{document}