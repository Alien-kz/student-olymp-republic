\documentclass[12pt, a4paper]{article}

\usepackage[T2A]{fontenc}
\usepackage[utf8]{inputenc}
\usepackage[english, russian]{babel}
\usepackage{amssymb}
\usepackage{amsfonts}
\usepackage{amsmath}
\usepackage{mathtext}

\usepackage{comment}
\usepackage{geometry}
\geometry{left=1cm, right=1cm, top=2cm, bottom=2cm}

\usepackage{graphicx}
\usepackage{tikz}

\usepackage{wrapfig}
\usepackage{fancybox,fancyhdr}
\sloppy

\setlength{\headheight}{28pt}

\newcommand{\head}[4]
{
	\fancyhf{}
	\pagestyle{fancy}
	\chead{#3, #4}

	\begin{center}
	\begin{large}
	#1 \\
	\end{large}
	#2 \\
	\end{center}

}

\begin{document}

\head{VI Республиканская студенческая предметная олимпиада по направлению \\ <<Математика>>}{27 марта 2014}{Казахский национальный университет имени аль-Фараби}{г. Алматы}

\begin{enumerate}
\item Вычислить предел:
$$\lim_{x \to +\infty} \frac{(x - \sqrt{x^2 - 1})^n + (x + \sqrt{x^2 - 1})^n}{x^n}.$$

\item Решить систему уравнений:
$$
\begin{cases}
\frac{x^2}{y^2} + \frac{y^2}{z^2} + \frac{z^2}{x^2} = 3 \\
x + y + z = 3 \\
\frac{y^2}{x^2} + \frac{z^2}{y^2} + \frac{x^2}{z^2} = 3 \\
\end{cases}
$$

\item Дан многочлен с целыми коэффициентами. В трех целых точках он принимает значение 2. Докажите, что ни в какой целой точке он не принимает значение 3.

\item Найти все решения дифференциального уравнения второго порядка $$f''(x) + f(\pi - x) = 1.$$

\end{enumerate}

\end{document} 
