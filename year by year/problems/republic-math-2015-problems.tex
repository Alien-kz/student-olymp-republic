\documentclass[12pt, a4paper]{article}

\usepackage[T2A]{fontenc}
\usepackage[utf8]{inputenc}
\usepackage[english, russian]{babel}
\usepackage{amssymb}
\usepackage{amsfonts}
\usepackage{amsmath}
\usepackage{mathtext}

\usepackage{comment}
\usepackage{geometry}
\geometry{left=1cm, right=1cm, top=2cm, bottom=2cm}

\usepackage{graphicx}
\usepackage{tikz}

\usepackage{wrapfig}
\usepackage{fancybox,fancyhdr}
\sloppy

\setlength{\headheight}{28pt}

\newcommand{\head}[4]
{
	\fancyhf{}
	\pagestyle{fancy}
	\chead{#3, #4}

	\begin{center}
	\begin{large}
	#1 \\
	\textit{#2} \\
	\end{large}
	\end{center}

}

\begin{document}

\head{VII Республиканская студенческая предметная олимпиада по направлению \\ <<Математика>>}{26 марта 2015}{Казахский национальный университет имени аль-Фараби}{г. Алматы}

\begin{enumerate}
\item Функция $f(x)$ дважды непрерывно дифференцируема на отрезке $[a, b]$ и имеет на $[a, b]$ не менее трех различных нулей. Докажите, что существует точка $x \in [a, b]$ такая, что $f(x) + f''(x) = 2 f'(x)$. 

\item Функция $f(x)$ дважды непрерывно дифференцируема на полуоси $[0, +\infty)$. Известно, что $f(x) > 0$, $f'(x) > 0$ и $\frac{f(x) f''(x)}{(f'(x))^2} \leqslant 2$ для всех $x \in [0, +\infty)$. Доказать, что $\lim_{x \to +\infty} \frac{f'(x)}{(f(x))^2} = 0$.

\item Известно, что для квадратных матриц $A$ и $B$ одинакового порядка выполнены следующие равенства $AB = BA$, $A^{99} = E$, $B^{100} = E$, где $E$ --- единичная матрица. Докажите, что существует матрица $C$ такая, что $(A + E + B) C = C (A + E + B) = E$.

\item Пусть
$$A = 
\begin{pmatrix}
3 & 2 \\
2 & 0 \\
\end{pmatrix},
A^n =
\begin{pmatrix}
a_{11}(n) & a_{12}(n) \\
a_{21}(n) & a_{22}(n) \\
\end{pmatrix}.
$$

Доказать существование и найти предел $\lim_{n \to \infty} \frac{a_{12}(n)}{a_{22}(n)}$.

\end{enumerate}

\end{document} 
