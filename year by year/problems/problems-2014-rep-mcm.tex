\documentclass[12pt, a4paper]{article}

\usepackage[T2A]{fontenc}
\usepackage[utf8]{inputenc}
\usepackage[english, russian]{babel}
\usepackage{amssymb}
\usepackage{amsfonts}
\usepackage{amsmath}
\usepackage{mathtext}

\usepackage{comment}
\usepackage{geometry}
\geometry{left=1cm, right=1cm, top=2cm, bottom=2cm}

\usepackage{graphicx}
\usepackage{tikz}

\usepackage{wrapfig}
\usepackage{fancybox,fancyhdr}
\sloppy

\setlength{\headheight}{28pt}

\newcommand{\head}[4]
{
	\fancyhf{}
	\pagestyle{fancy}
	\chead{#3, #4}

	\begin{center}
	\begin{large}
	#1 \\
	\end{large}
	#2 \\
	\end{center}

}

\begin{document}

\head{VI Республиканская студенческая предметная олимпиада по направлению \\ <<Математическое и компьютеное моделирование>>}{27 марта 2014}{Казахский национальный университет имени аль-Фараби}{г. Алматы}

\begin{flushright}
Стоимость задач: \\
10, 20, 25, 15 и 20 баллов соответственно.\\
\end{flushright}

\begin{enumerate}
\item Исследовать на сходимость ряд $\sum\limits_{n=1}^{\infty} a_n$, общий член которого имеет вид 
$$a_n = \left( \frac{1 \cdot 3 \cdot 5 \cdot \ldots \cdot (2n - 1)}{2 \cdot 4 \cdot 6 \cdot \ldots \cdot 2n} \right)^2.$$

\item Рассмотрим обыкновенное дифференциальное уравнение Клеро $y = p \cdot x + f(p)$, где $p = y'$ и $f$ --- дифференцируемая функция. Доказать, что если $f'$ --- монотонная функция, то особое решение уравнения Клеро имеет ровно одну общую точку с любым частным решением.

\item Во время недавних раскопок на Марсе были обнаружены листы бумаги с таинственными символами на них. После долгих исследований учёные пришли к выводу, что надписи на них на самом деле могли быть обычными числовыми равенствами. Кроме того, из других источников было получено веское доказательство того, что марсиане знали только три операции: сложение, умножение и вычитание (марсиане никогда не использовали <<унарный минус>>: вместо <<$-5$>>они писали <<$0-5$>>). Также ученые доказали, что марсиане не наделяли операции разным приоритетом, а просто вычисляли выражения (если в них не было скобок) слева направо: например, $3+3*5$ у них равнялось 30, а не 18. К сожалению, символы арифметических действий стерлись. Например, если была запись <<$18=7 (5 3) 2$>>, то возможно восстановить эту запись как <<$18=7+(5-3)*2$>>. Требуется написать программу, находящую требуемую расстановку знаков или сообщающую, что таковой не существует.

Первая строка входного файла {\tt INPUT.TXT} состоит из натурального числа, не превосходящего $2^{30}$, знака равенства, и последовательности натуральных чисел (не более десяти), произведение которых также не превосходит $2^{30}$. Некоторые группы чисел (одно или более) могут быть окружены скобками. Длина входной строки не будет превосходить 80 символов, и других ограничений на количество и вложенность скобок нет. Между двумя соседними числами, не разделенными скобками, всегда будет хотя бы один пробел, во всех остальных местах может быть любое (в том числе и 0) число пробелов (естественно, внутри числа пробелов нет).

В выходной файл {\tt OUTPUT.TXT} необходимо вывести одну строку, содержащую полученное равенство (т.е., исходное равенство со вставленными знаками арифметических действий без лишних пробелов). В случае если требуемая расстановка знаков невозможна, вывести строку, состоящую из единственного числа <<$-1$>>. Выходная строка не должна содержать пробелов.

\item Вычислите определитель

$$
\left| \begin{array}{cccc}
P(x) & P(x+1) & \dots & P(x+2014) \\
P'(x) & P'(x+1) & \dots & P'(x+2014) \\
\dots & \dots & \dots & \dots \\
P^{(2014)}(x) & P^{(2014)}(x+1) & \dots & P^{(2014)}(x+2014) \\
\end{array}\right|,
$$
где $P(x) = x (x+1) \cdot \ldots \cdot (x + 2014)$.

\item Внутри тетраэдра $ABCD$ расположена точка $O$ так, что прямые $AO$, $BO$, $CO$, $DO$ пересекают грани $BCD$, $ACD$, $ABD$, $ABC$ тетраэдра в точках $A_1$, $B_1$, $C_1$, $D_1$ соответственно. Причем отношения 
$$
\frac{AO}{A_1O}, 
\frac{BO}{B_1O},
\frac{CO}{C_1O},
\frac{DO}{D_1O}
$$
равны одному и тому же числу. Найти все значения, которые может принимать это число.
\end{enumerate}

\end{document} 
