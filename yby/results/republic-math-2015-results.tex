\documentclass[12pt, a5paper, landscape]{article}

\usepackage[T2A]{fontenc}		%cyrillic output
\usepackage[utf8]{inputenc}		%cyrillic output
\usepackage[english, russian]{babel}	%word wrap
\usepackage{amssymb, amsfonts, amsmath}	%math symbols
\usepackage{mathtext}			%text in formulas
\usepackage{geometry}			%paper format attributes
\usepackage{fancyhdr}			%header
\usepackage{graphicx}			%input pictures
\usepackage{tikz}				%draw pictures
\usetikzlibrary{patterns}		%draw pictures: fill

\geometry{left=1cm, right=1cm, top=2cm, bottom=1cm, headheight=15pt}
\sloppy							%correct overfull

\newcommand{\head}[4]
{
	\pagestyle{fancy}
	\fancyhf{}
	\chead{#3, #4}

	\begin{center}
	\begin{large}
	#1 \\
	\textit{#2}\\
	\end{large}
	\end{center}

}

\begin{document}

\head{VII Республиканская студенческая предметная олимпиада по~направлению~<<Математика>>}{26 марта 2015}{Казахский национальный университет имени аль-Фараби}{г.~Алматы}

\begin{center}
\begin{tabular}{|l|l|l|c|c|c|c|c|c|c|}
\hline
№ & Участник & ВУЗ & Курс & Диплом \\
\hline
1 & Шокетаева Надира & КФ МГУ & 2 & 1 \\
\hline
2 & Журавская Александра &  КФ МГУ & 1 & 2 \\
\hline
3 & Тыныштыкбай Абылай & КарГУ & 3 & 2 \\
\hline
4 & Джумагулов Серик & КарГУ & ? & 3 \\
\hline
5 & Кабак Мухтар & КазНУ & 3 & 3 \\
\hline
6 & Таскынов Ануар & КФ МГУ & 2 & 3 \\
\hline
\end{tabular}
\end{center}

%http://www.kaznu.kz/content/files/news/folder14971/%D0%9E%D0%BB%D0%B8%D0%BC%D0%BF%D0%B8%D0%B0%D0%B4%D0%B0_%D0%BE%D1%82%D1%87%D0%B5%D1%82%202015.doc
%http://student.ksu.kz/novosti/student-fakulteta-matematiki-i-informacionnyh-tehnologii-kargu-im-e-a-buketova-tynyshtykbai-abylai-zanjal-2-e-i-dzhumagulov-serik-zanjal-3-e-mesto-v-lichnom-zachete-na-vii-respublikanskoi-stud.html

\end{document} 
