\documentclass[11pt, a5paper]{article}

\usepackage[T2A]{fontenc}
\usepackage[utf8]{inputenc}
\usepackage[english, russian]{babel}
\usepackage{amssymb}
\usepackage{amsfonts}
\usepackage{amsmath}
\usepackage{mathtext}

\usepackage{comment}
\usepackage{geometry}
\geometry{left=0.5cm, right=1cm, top=1cm, bottom=1cm}
\usepackage[inline]{enumitem}

\usepackage{graphicx}
\usepackage{tikz}
\usetikzlibrary{patterns}

\usepackage{wrapfig}
\usepackage{fancybox,fancyhdr}
\sloppy

\setlength{\headheight}{28pt}

\newcommand{\head}[4]
{
	\fancyhf{}
	\pagestyle{fancy}
	\chead{#3, #4}

	\begin{center}
	\begin{large}
	#1 \\
	\textit{#2}\\
	\end{large}
	\end{center}

}

\begin{document}

\head{Республиканская студенческая олимпиада по~математике}{2006}{Актюбинский региональный государственный университет}{г. Актобе}

\begin{enumerate}
\item (6 баллов) Пусть $f(x) \in C[0, +\infty)$ и для любого $x \geqslant 0$ справедливо равенство
$$ \sin \int_{0}^{x} f(u) du = \frac{x}{x+1}$$
Определить функцию $f(x)$.

\item (8 баллов) Доказать, что для любого $n \in N \cup \{0\}$ справедливо равенство $$\int_{0}^{2 \pi} \sin(\sin x + n x) dx = 0.$$

\item (8 баллов) Доказать, что если $\overrightarrow{AA_1} + 
\overrightarrow{BB_1} +
\overrightarrow{CC_1} = \overrightarrow{0}$, где $[AA_1]$, $[BB_1]$, $[CC_1]$ --- биссекторисы треугольника, то треугольник правильный.

\item (8 баллов) Найти наибольшее значение определителя 3-го порядка, элементы которого состоят из 1 и $-1$.
\end{enumerate}

\end{document} 
