\documentclass[11pt, a4paper]{article}

\usepackage[T2A]{fontenc}
\usepackage[utf8]{inputenc}
\usepackage[english, russian]{babel}
\usepackage{amssymb}
\usepackage{amsfonts}
\usepackage{amsmath}
\usepackage{mathtext}

\usepackage{comment}
\usepackage{geometry}
\geometry{left=0.5cm, right=1cm, top=1cm, bottom=1cm}
\usepackage[inline]{enumitem}

\usepackage{graphicx}
\usepackage{tikz}
\usetikzlibrary{patterns}

\usepackage{wrapfig}
\usepackage{fancybox,fancyhdr}
\sloppy

\setlength{\headheight}{28pt}
\newcommand{\variant}[2]{
	\begin{center}
	\textit{Вариант #2}
	\end{center}
}

\newcommand{\unit}[1]{\text{\textit{ #1}}}
\newcommand{\units}[2]{ \frac{\text{\textit{#1}}}{\text{\textit{#2}}}}

\newcommand{\head}[4]
{
	\fancyhf{}
	\pagestyle{fancy}
	\chead{#3, #4}

	\begin{center}
	\begin{large}
	#1 \\
	\textit{#2}\\
	\end{large}
	\end{center}

}

\begin{document}

\head{IX Республиканская студенческая предметная олимпиада по~направлению~<<Математическое и компьютерное моделирование>>}{14 апреля 2017}{Назарбаев Университет}{г. Астана}

\begin{enumerate}
\item Назовем натуральное число $a$ <<хорошим>>, если его можно представить в виде $a = 10^e (1+m)$, где $e$ и $m$ такие целые числа, что $0 \leqslant e < 224$ и $0 \leqslant m < 10^{100}$.

a) Найдите минимальное натуральное число, которое не является <<хорошим>>;

б) Найдите количество <<хороших>> натуральное чисел.

\item Сваха собрала базу данных из $n$ мужчин и $n$ женщин и желает их всех
переженить. Проблема осложняется тем, что не все пары <<мужчина --- женщина>> психологически совместимы (то есть, поженить такую пару нельзя). Назовем любовным циклов список из $k > 2$ различных персон $X_1$, $\dots$, $X_k$, таких, что $X_1$ совместим с $X_2$, $X_2$ с $X_3$, $\dots$, $X_{k-1}$ с $X_k$, $X_k$ с $X_1$. Собрав всю информацию о взаимной совместимости, сваха обнаружила, что в ее базе всего лишь один любовный цикл. Обозначим $N$ общее количество способов которыми сваха сможет поженить всех. Два способа считаются различными, если хотя бы один человек в них вступает в брак с разными партнерами. Какие значения может принимать $N$?

\item В тридевятом царстве в тридесятом государстве жил был король-самодур и математик при его дворе. В государстве было $n$ городов и длины дорог между ними были известны (они принимали натуральные значения и были не длиннее 100 км). Король попросил математика найти длину наикратчайшего пути между западной и восточной столицами с допустимой ошибкой не более 1\%, пригрозив при этом казнью за любое сравнение чисел (операции $\min{x, y}$ и $\max{x, y}$ тоже запрещены). Разрешены лишь 
\begin{itemize}
\item суммирование;
\item умножение;
\item деление;
\item возведение в степень;
\item вычисление логарифма.
\end{itemize}
Как быть математику? Опишите его алгоритм (как можно более оптимальный по вычислительным затратам). Оцените вычислительную сложность этого алгоритма (то есть, получите верхнюю оценку $g(n)$ на общее количество вышеперечисленных операций в данном алгоритме).

\item Вам дали задание написать программу которая должна делать следующее:
\begin{itemize}
\item вначале прочитать файл с n вещественными числами $a_1$, $a_2$, $\dots$, $a_n$ и создать массив $A[i] = a_i$, $1 \leqslant i \leqslant n$.
\item если вы подаете ей на вход пару $(i, j)$ такую, что $1 \leqslant i \leqslant j \leqslant n$, то
\end{itemize}
программа должна возвратить $\max_{i \leqslant k \leqslant j} a_k$. Ваша программа, таким образом, существует в двух режимах:
\begin{enumerate}
\item в режиме препроцессинга массива $A$ --- в этом режиме вы можете обработать массив $A$, создать какие-то структуры данных, чтобы более эффективно выполнить работу на втором этапе; 
\item в режиме расчета для заданной пары $(i, j)$ ответа $\max_{i \leqslant k \leqslant j} a_k$.
\end{enumerate}
Опишите алгоритм, в котором
\begin{itemize}
\item Первый этап занимает $O(n^2)$ арифметических операций;
\item Второй этап занимает $O(1)$ арифметических операций;
\end{itemize}
Опишите алгоритм, в котором
\begin{itemize}
\item Первый этап занимает $O(n \log n)$ арифметических операций
\item Второй этап занимает $O(1)$ арифметических операций
\end{itemize}

\end{enumerate}


\end{document} 
