\documentclass[11pt, a4paper]{article}

\usepackage[T2A]{fontenc}
\usepackage[utf8]{inputenc}
\usepackage[english, russian]{babel}
\usepackage{amssymb}
\usepackage{amsfonts}
\usepackage{amsmath}
\usepackage{mathtext}

\usepackage{comment}
\usepackage{geometry}
\geometry{left=0.5cm, right=1cm, top=1cm, bottom=1cm}
\usepackage[inline]{enumitem}

\usepackage{graphicx}
\usepackage{tikz}
\usetikzlibrary{patterns}

\usepackage{wrapfig}
\usepackage{fancybox,fancyhdr}
\sloppy

\setlength{\headheight}{28pt}
\newcommand{\variant}[2]{
	\begin{center}
	\textit{Вариант #2}
	\end{center}
}

\newcommand{\unit}[1]{\text{\textit{ #1}}}
\newcommand{\units}[2]{ \frac{\text{\textit{#1}}}{\text{\textit{#2}}}}

\newcommand{\head}[4]
{
	\fancyhf{}
	\pagestyle{fancy}
	\chead{#3, #4}

	\begin{center}
	\begin{large}
	#1 \\
	\textit{#2}\\
	\end{large}
	\end{center}

}

\begin{document}

\head{X Республиканская студенческая предметная олимпиада по~направлению~<<Математическое и компьютерное моделирование>>}{20 апреля 2018}{Назарбаев Университет}{г. Астана}

\begin{flushright}
Стоимость задач: \\
7 баллов каждая задача.\\
\end{flushright}

\begin{enumerate}
\item Мяч весом $0,1$ килограмма подбрасывают вверх с земли с начальной скоростью $20$ м/с. Сила сопротивления воздуха величиной $|v|^3/1020,4$ направлена в сторону противоположную скорости мяча, где скорость v измеряется в м/с. Найдите формулу для вычисления времени, которое требуется мячу для достижения максимальной высоты над уровнем земли. 

\item Допустим $1 < p < 2$ и дана функция $\varphi(x) = |x|^{p-2} x$, где $x \in \mathbb{R}$. Докажите неравенство
$$|\varphi(x) - \varphi(y)| \leqslant 2^{2-p} |x-y|^{p-1}.$$
Также покажите, что равенство имеет место если $y = -x$.

\item Вычислите точное значение $\cos\left(\frac{\pi}{10}\right)$.

\item Вычислите интеграл
$$\int_{0}^{1} \sqrt{-\ln(x)} dx.$$

\item Даны 52 точки $P_1(x_1, y_1)$, $P_2(x_2, y_2)$, \ldots, $P_{52}(x_{52}, y_{52})$, расположенные в квадрате, длина стороны которого равна 7. Покажите, что среди этих 52 точек всегда можно найти 3 точки,
которые расположены внутри круга радиуса 1.
\end{enumerate}

\end{document} 
